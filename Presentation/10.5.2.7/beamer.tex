\documentclass{beamer}
\usetheme{Madrid}
\usepackage{minted}
\usemintedstyle{friendly}
\definecolor{bg}{rgb}{0.95,0.95,0.95}
\usepackage{amsmath, amssymb, amsthm}
\usepackage{graphicx}
\usepackage{listings}
\usepackage{gensymb}
\usepackage[utf8]{inputenc}
\usepackage{hyperref}
\usepackage{gvv}
%\usepackage{mihir}
% \pagestyle{empty}% for cropping
% \usepackage{minted,newtxtext,newtxmath,caption}
% \newcommand\code{\texttt}
% \newcommand\param{\textit}
% \DeclareCaptionFont{bfmath}{\boldmath\bfseries}
% \DeclareCaptionFormat{ruled}{\hrulefill\par#1#2#3}
% \captionsetup{format=ruled,font=bfmath}
\begin{document}

\title{NCERT 10.5.2.7}
\author{EE23BTECH11017 - Eachempati Mihir Divyansh$^{*}$}
\date{}
\frame{\titlepage}
\begin{frame}   
\frametitle{Question}
Find the 31st term of an AP whose $11$th term is 38 and the $16$th term is 73.
\end{frame}
\begin{frame}{allowframebreaks}
\frametitle{Given Informatn}
\begin{table}
    \centering
   \begin{tabular}{|m{5em}|m{5em}| m{10em} | }
    \hline
    \textbf{Symbol} &\textbf{Value} &\textbf{Description} \\
    \hline
         $x\brak{0}$ & -32 & First term  \\
    \hline
        $x\brak{10}$ & 38  & $11$th term \\
    \hline
        $x\brak{15}$ & 73 & $16$th term\\
    \hline
        $d$ & 7 & Common Difference\\
    \hline
        $x\brak{n}$ & $x(0)+nd$ & $\brak{n+1}$th term\\
    \hline
    \end{tabular} 
        \caption{Given Information}
        \label{tab:1}
    \end{table}
\end{frame}
\begin{frame}{allowframebreaks}
\frametitle{Solution: Part 1}
From \tabref{tab:1} 
\begin{align}
x\brak{0}+10d&=38\label{10.5.2.7.eq: 1}\\
x\brak{0}+15d&=73 \label{10.5.2.7.eq: 2}
\end{align}
From  equations \ref{10.5.2.7.eq: 1} and \ref{10.5.2.7.eq: 2}, the augmented matrix is:
\begin{align}
 \myvec{
   1 & 10 & 38
   \\
   1 & 15 & 73
 } 
\end{frame}
\begin{frame}{allowframebreaks}
    \frametitle{Expressing $x(0)$ and $r$ in terms of $x$ and $y$}
    From \eqref{eqn: 2}, 
    \begin{align}
        r^6=\frac{y}{x}\;\;
        &\implies r=\sqrt[6]{\frac{y}{x}}=\brak{\dfrac{y}{x}}^{\frac{1}{6}}\\
        x\brak{0}=\frac{x}{r^3}\;\;
        &\implies x\brak{0}=\brak{\dfrac{x^3}{y}}^{\frac{1}{2}}
    \end{align}
    Z-Transform of $x\brak{0}$
    \begin{align}
        X\brak{z}&=\frac{x\brak{0}}{1-rz^{-1}}\\
        &=\dfrac{\brak{\frac{x^3}{y}}^{\frac{1}{2}}}{1-\brak{\frac{y}{x}}^{\frac{1}{6}}z^{-1}}
    \end{align}
\end{frame}
\begin{frame}{allowframebreaks}
    \frametitle{Exaample}
    Let $x(0)=1$ and $r=1.2$
\begin{align}
    x=x\brak{3}=& \brak{1.2}^3 \\
    y=x\brak{9}=&\brak{1.2}^9\\
    z=x\brak{15}=&\brak{1.2}^{15}
\end{align}
\end{frame}
\begin{minted}[bgcolor=bg, linenos, fontsize=\small, breaklines, language=c, lexer=minted-clang]{c}
    #include<stdio.h>
#include<math.h>

int main(){
    FILE *ptr;
    ptr= fopen("series.dat", "w");
    float x_0=1.0;
    float r= 1.2;
    for(int i=0; i<17; i++){
        fprintf(ptr, "%f ", x_0*pow(r,i));
    }
    fprintf(ptr, "\b ");
    return 0;
}
\end{minted}
\end{frame}
\end{document}
