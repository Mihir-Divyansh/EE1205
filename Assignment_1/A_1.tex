\let\negmedspace\undefined
\let\negthickspace\undefined
\documentclass[journal,12pt,twocolumn]{IEEEtran}
\usepackage{cite}
\usepackage{amsmath,amssymb,amsfonts,amsthm}
\usepackage{algorithmic}
\usepackage{graphicx}
\usepackage{textcomp}
\usepackage{xcolor}
\usepackage{txfonts}
\usepackage{listings}
\usepackage{enumitem}
\usepackage{mathtools}
\usepackage{gensymb}
\usepackage{comment}
\usepackage[breaklinks=true]{hyperref}
\usepackage{tkz-euclide} 
\usepackage{listings}
\usepackage{gvv}                                        
%\def\inputGnumericTable{}                                 
\usepackage[latin1]{inputenc}                                
\usepackage{color}                                            
\usepackage{array}                                            
\usepackage{longtable}                                       
\usepackage{calc}                                             
\usepackage{multirow}                                         
\usepackage{hhline}                                           
\usepackage{ifthen}                                           
\usepackage{lscape}
\usepackage{tabularx}
\usepackage{array}
\usepackage{float}


\newtheorem{theorem}{Theorem}[section]
\newtheorem{problem}{Problem}
\newtheorem{proposition}{Proposition}[section]
\newtheorem{lemma}{Lemma}[section]
\newtheorem{corollary}[theorem]{Corollary}
\newtheorem{example}{Example}[section]
\newtheorem{definition}[problem]{Definition}
\newcommand{\BEQA}{\begin{eqnarray}}
\newcommand{\EEQA}{\end{eqnarray}}
\newcommand{\define}{\stackrel{\triangle}{=}}
\theoremstyle{remark}
\newtheorem{rem}{Remark}
\begin{document}

\bibliographystyle{IEEEtran}
\vspace{3cm}

\title{11.9.3.17}
\author{EE23BTECH11017 - Eachempati Mihir Divyansh$^{*}$% <-this % stops a space
}
\maketitle
\newpage
\bigskip

\renewcommand{\thefigure}{\theenumi}
\renewcommand{\thetable}{\theenumi}

\textbf{Question: }
If the $4^{th}$, $10^{th}$ and $16^{th}$ terms of a G.P. are x, y, and z, respectively. Prove that $x,\; y,\; z$ are in G.P.

\begin{table}[h]
    \renewcommand\thetable{1}
    \centering
        \caption{\textbf{Given Information}}
    \begin{tabular}{|m{2cm}|m{2cm}|m{2cm}|}
    \hline
    \textbf{Symbol} & \textbf{Value} & \textbf{Description}\\ [1ex]
    \hline
        $x$ & $x(0)r^4$ & $x(4)$ \\ [1ex]
    \hline
        $y$ & $x(0)r^{10}$ & $x(10)$\\ [1ex]
    \hline
        $z$ & $x(0)r^{16}$ & $x(16)$\\ [1ex]
    \hline
        $r$ & $y^{\frac{1}{6}}x^{-\frac{1}{6}}$ & $\frac{x(n)}{x(n-1)}$\\[1ex]
    \hline \vspace{0.1cm}
        $x(0)$ & $x^{\frac{5}{3}}y^{-\frac{2}{3}}$ & First term \\[1ex]
    \hline
        $x(n)$ & $x(0)r^nu(n)$ & General Term \\ [1ex]
    \hline
    \end{tabular}\label{Table 1}
\end{table} 

\solution

From \tabref{Table 1},
\begin{align}
\notag x&= x(4) =x(0)r^4 \\
\notag y&=x(10)=x(0)r^{10} \\
\notag z&=x(16)=x(0)r^{16}
\end{align}
Consider $\dfrac{x(10)}{x(4)}$ and $\dfrac{x(16)}{x(10)}$;
\begin{align}
 \dfrac{x(10)}{x(4)} &= \dfrac{x(0)r^{10}}{x(0)r^4} = r^6\\ 
 \dfrac{x(16)}{x(10)} &= \dfrac{x(0)r^{16}}{x(0)r^{10}} = r^6
\end{align}
Since, $\dfrac{x(10)}{x(4)}$ = $\dfrac{x(16)}{x(10)}$;\\
\begin{align}  
\notag   & x(4),\; x(10),\; x(16)\text{ are in G.P.} \\
\notag  \therefore & x,\; y,\; z \text{ are in G.P.}
\end{align}


To extend the domain of n to -ve integers, the step function $u(n)$ can be used.
\begin{align}
 \notag    \therefore x(n) &= x(0)r^n u(n) \; \forall \;n\in Z
\end{align}
$x(0)$ $r$ can be expressed in terms of $x$, $y$, and $z$ in the following manner.
\begin{align}
 \notag      x&=x(0)r^4 \\
 \notag    \frac{y}{x}&=r^6 \\
 \Rightarrow r&=\sqrt[6]{\frac{y}{x}}=(\frac{y}{x})^{\frac{1}{6}}\\
 \notag    x(0)&=\frac{x}{r^4}\\
 \notag    x(0)&=x(\frac{x}{y})^{\frac{4}{6}}\\
 \therefore\; x(0)&=x^{\frac{5}{3}}y^{-\frac{2}{3}} \\
 and\; r&=(\frac{y}{x})^{\frac{1}{6}}= y^{\frac{1}{6}}x^{-\frac{1}{6}}
\end{align}





\end{document}
