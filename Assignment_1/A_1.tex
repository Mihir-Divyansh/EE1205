\let\negmedspace\undefined
\let\negthickspace\undefined
\documentclass[journal,12pt,twocolumn]{IEEEtran}
\usepackage{cite}
\usepackage{amsmath,amssymb,amsfonts,amsthm}
\usepackage{algorithmic}
\usepackage{graphicx}
\usepackage{textcomp}
\usepackage{xcolor}
\usepackage{txfonts}
\usepackage{listings}
\usepackage{enumitem}
\usepackage{mathtools}
\usepackage{gensymb}
\usepackage{comment}
\usepackage[breaklinks=true]{hyperref}
\usepackage{tkz-euclide}
\usepackage{listings}
\usepackage{gvv}                                        
%\def\inputGnumericTable{}                                
\usepackage[latin1]{inputenc}                                
\usepackage{color}                                            
\usepackage{array}                                            
\usepackage{longtable}                                      
\usepackage{calc}                                            
\usepackage{multirow}                                        
\usepackage{hhline}                                          
\usepackage{ifthen}                                          
\usepackage{lscape}
\usepackage{tabularx}
\usepackage{array}
\usepackage{float}

\newtheorem{theorem}{Theorem}[section]
\newtheorem{problem}{Problem}
\newtheorem{proposition}{Proposition}[section]
\newtheorem{lemma}{Lemma}[section]
\newtheorem{corollary}[theorem]{Corollary}
\newtheorem{example}{Example}[section]
\newtheorem{definition}[problem]{Definition}
\newcommand{\BEQA}{\begin{eqnarray}}
\newcommand{\EEQA}{\end{eqnarray}}
\newcommand{\define}{\stackrel{\triangle}{=}}
\theoremstyle{remark}
\newtheorem{rem}{Remark}
\begin{document}

\bibliographystyle{IEEEtran}
\vspace{3cm}

\title{11.9.3.17}
\author{EE23BTECH11017 - Eachempati Mihir Divyansh$^{*}$% <-this % stops a space
}
\maketitle
\newpage
\bigskip

\renewcommand{\thefigure}{\theenumi}
\renewcommand{\thetable}{\theenumi}

\textbf{Question: }
If the $4^{th}$, $10^{th}$ and $16^{th}$ terms of a G.P. are x, y, and z, respectively. Prove that $x,\; y,\; z$ are in G.P.

\begin{table}[h]
    \renewcommand\thetable{1}
    \centering
        \caption{\textbf{Given Information}}
    \begin{tabular}{|m{2cm}|m{2cm}|m{2cm}|}
    \hline
    \textbf{Symbol} & \textbf{Value} & \textbf{Description}\\ [1ex]
    \hline
        $x$ & $ar^4$ & $x(4)$ \\ [1ex]
    \hline
        $y$ & $ar^{10}$ & $x(10)$\\ [1ex]
    \hline
        $z$ & $ar^{16}$ & $x(16)$\\ [1ex]
    \hline
        $a$ & $x^{\frac{5}{3}}y^{-\frac{2}{3}}$ & x(0) \\[1ex]
    \hline
        $r$ & $y^{\frac{1}{6}}x^{-\frac{1}{6}}$ & $\frac{x(n)}{x(n-1)}$\\[1ex]
    \hline \vspace{0.1cm}
        $x(n)$ & $y^{n-1} x^{2-n} u(n)$ & General Term \\ [1ex]
    \hline
    \end{tabular}
\end{table} \label{table}

\solution


The n$^{th}$ term of a G.P. is $a_n=ar^n$. Given that $x,\; y,\; z$ are the $4^{th}$, $10^{th}$ and $16^{th}$ terms of a G.P., From the \tabref{table},
\begin{align}
\notag x&= a_4 =ar^4 \\
\notag y&=a_{10}=ar^{10} \\
\notag z&=a_{16}=ar^{16}
\end{align}
Consider $\dfrac{y}{x}$ and $\dfrac{z}{y}$;
\begin{align}
\dfrac{y}{x} &= \dfrac{ar^{10}}{ar^4} = r^6\\
\dfrac{z}{y} &= \dfrac{ar^{16}}{ar^{10}} = r^6
\end{align}
Since, $\dfrac{y}{x}$ = $\dfrac{z}{y}$;\\
\begin{align}  
\notag    x,\; y,\; z \text{ are in G.P.}
\end{align}

For this G.P, with $x,\;y,\;z$, as the first three terms, the general term $x(n)$ can be defined as:

\begin{align}
\notag \text{Common Ratio}&=\frac{y}{x} \\
x(n)&=x({\frac{y}{x}})^{n-1} \\
also,\: x(n)&=x\cdot (\frac{z}{y})^{n-1} \\
\notag \therefore x(n)&=\frac{y^{n-1}}{x^{n-2}}\; \forall\; n\geq0
\end{align}
To extend the domain of n to -ve integers, the step function $u(n)$ can be used.
\begin{align}
\notag    \therefore x(n) &= \frac{y^{n-1}}{x^{n-2}} u(n) \; \forall \;n\in Z
\end{align}
The initial term $x(0)$ is :
\begin{align}
    x(0)&=x(n)/r^n\\
\notag    &=\brak{y^{n-1} x^{2-n} u(n)}\brak{\frac{y}{x}}^{-n}\\
    x(0)&=\frac{x^2}{y}
\end{align}

$a$ and $r$ can be expressed in terms of $x$, $y$, and $z$ in the following manner.
\begin{align}
\notag      x&=ar^4 \\
\notag    \frac{y}{x}&=r^6 \\
\Rightarrowr&=\sqrt[6]{\frac{y}{x}}=(\frac{y}{x})^{\frac{1}{6}}\\
\notag    a&=\frac{x}{r^4}\\
\notag    a&=x(\frac{x}{y})^{\frac{4}{6}}\\
\therefore a&=x^{\frac{5}{3}}y^{-\frac{2}{3}} \\
and\; r&=(\frac{y}{x})^{\frac{1}{6}}= y^{\frac{1}{6}}x^{-\frac{1}{6}}
\end{align}





\end{document}

 
