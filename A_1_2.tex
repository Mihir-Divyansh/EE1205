
% \iffalse
\let\negmedspace\undefined
\let\negthickspace\undefined
\documentclass[journal,12pt,twocolumn]{IEEEtran}
\usepackage{cite}
\usepackage{amsmath,amssymb,amsfonts,amsthm}
\usepackage{algorithmic}
\usepackage{graphicx}
\usepackage{textcomp}
\usepackage{xcolor}
\usepackage{txfonts}
\usepackage{listings}
\usepackage{enumitem}
\usepackage{mathtools}
\usepackage{gensymb}
\usepackage{comment}
\usepackage[breaklinks=true]{hyperref}
\usepackage{tkz-euclide} 
\usepackage{listings}
\usepackage{gvv}                                        
\def\inputGnumericTable{}                                 
\usepackage[latin1]{inputenc}                                
\usepackage{color}                                            
\usepackage{array}                                            
\usepackage{longtable}                                       
\usepackage{calc}                                             
\usepackage{multirow}                                         
\usepackage{hhline}                                           
\usepackage{ifthen}                                           
\usepackage{lscape}

\newtheorem{theorem}{Theorem}[section]
\newtheorem{problem}{Problem}
\newtheorem{proposition}{Proposition}[section]
\newtheorem{lemma}{Lemma}[section]
\newtheorem{corollary}[theorem]{Corollary}
\newtheorem{example}{Example}[section]
\newtheorem{definition}[problem]{Definition}
\newcommand{\BEQA}{\begin{eqnarray}}
\newcommand{\EEQA}{\end{eqnarray}}
\newcommand{\define}{\stackrel{\triangle}{=}}
\theoremstyle{remark}
\newtheorem{rem}{Remark}
\begin{document}

\bibliographystyle{IEEEtran}
\vspace{3cm}

\title{11.9.3.17}
\author{EE23BTECH11017 - Eachempati Mihir Divyansh$^{*}$% <-this % stops a space
}
\maketitle
\newpage
\bigskip

\renewcommand{\thefigure}{\theenumi}
\renewcommand{\thetable}{\theenumi}

\textbf{Question: }
If the $4^{th}$, $10^{th}$ and $16^{th}$ terms of a G.P. are x, y, and z, respectively. Prove that $x,\; y,\; z$ are in G.P.\\
\solution

The n$^{th}$ term of a G.P. is $a_n=a_1r^{n-1}$. Given that $x,\; y,\; z$ are the $4^{th}$, $10^{th}$ and $16^{th}$ terms of a G.P.,
\begin{align}
\notag    x = a_4 =ar^{4-1}=ar^3 \\
\notag     y=a_{10}=ar^{10-1}=ar^9 \\
\notag     z=a_{16}=ar^{16-1}=ar^{15}
\end{align}

Consider $\dfrac{y}{x}$ and $\dfrac{z}{y}$;

\begin{align}
    \dfrac{y}{x} = \dfrac{ar^9}{ar^3} \\
    \dfrac{y}{x} = r^6\\ \notag\\
    \dfrac{z}{y} = \dfrac{ar^{15}}{ar^9} \\ 
    \dfrac{y}{x} = r^6
\end{align}

Since, $\dfrac{y}{x}$ = $\dfrac{z}{y}$;\\
\begin{align}  
\notag    x,\; y,\; z \text{ are in G.P.}
\end{align}
\end{document}
